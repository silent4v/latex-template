% !TeX root = main.tex
% === 功能性套件 ===
\usepackage{etoolbox} % 提供 pretocmd 等條件與補丁指令
\usepackage{chngcntr} % 控制計數器(如圖表獨立編號)

% === 字體、顏色與排版輔助 ===
\usepackage{epigraph} % 引用文字
\usepackage{fontspec} % 系統字型使用
\usepackage{xcolor}   % 顏色設置
\usepackage{ragged2e} % 文本對齊控制(如 \justifying)
\usepackage{xeCJK}    % CJK 字體相關
%% 該網址包含 Overleaf 支援的字體
% https://www.overleaf.com/learn/latex/Questions/Which_OTF_or_TTF_fonts_are_supported_via_fontspec
\setmainfont{Times New Roman}
\color{black}       % 設置字體顏色為黑色

% === 版面與格式調整 ===
\usepackage[a4paper]{geometry} % 邊界設定
\usepackage{setspace}          % 行距設定
\usepackage{titlesec}          % 標題格式控制
\usepackage{titletoc}          % 目錄樣式控制
\usepackage{fancyhdr}          % 頁眉頁腳設置
\usepackage{zhnumber}          % 中文數字格式(章節用)
\usepackage{background}        % 背景控制

%% 題目與標題空間設置
\titleformat{\chapter}[display]{
  \bfseries\fontsize{24pt}{24pt}\selectfont}
  {\chaptername\ \thechapter}{20pt}
  {\bfseries\fontsize{20pt}{24pt}\selectfont}
\titlespacing*{\chapter}{0pt}{-20pt}{40pt} 

%% 浮水印
\backgroundsetup{
  scale=0.8,
  angle=0,
  opacity=0.05,
  contents={\includegraphics{ntust}}
}

%% 版型
\newgeometry{
  top=3cm,
  left=3cm,
  right=3cm,
  bottom=2cm
}

\onehalfspacing % 設置行距為 1.5 倍行高

%% 每個章節 Counter Reset
\makeatletter
\@addtoreset{section}{chapter}
\@addtoreset{subsection}{section}
\@addtoreset{subsubsection}{subsection}
\@addtoreset{paragraph}{subsubsection}
\pretocmd{\chapter}{\setcounter{section}{0}}{}{}

%% 頁碼設定
\pagestyle{fancy}
\fancyhf{}
\fancyfoot[C]{\thepage} % 頁碼置於下方中間

%% 目錄的格式
\titlecontents{tableofcontents}
  [0pt]                              % 縮排
  {\bfseries}                        % above code
  {\contentslabel{2em}}              % numbered entry format
  {}                                 % numberless entry format
  {\titlerule*[1pc]{.}\contentspage} % filler and page number format

% === 數學與符號 ===
\usepackage{ amsmath, amssymb} % 高級數學符號

% === 表格處理 ===
\usepackage{longtable}  % 跨頁表格
\usepackage{booktabs}   % 表格橫線
\usepackage{multirow}   % 表格合併行
\usepackage{array}      % 表格設定
\usepackage{tabularx}   % 自動調整欄寬的表格
\usepackage{pifont}     % for checkmark
% === 列表與項目編排 ===
\usepackage{enumitem}   % 列表項目格式

% === 浮動體控制與圖文插入 ===
\usepackage{float}      % 浮動環境控制
\usepackage{graphicx}   % 插入圖片
\usepackage{subcaption} % 圖片的子圖顯示
\usepackage{longtable} % 長表格
\usepackage{caption}    % 圖片標題設定
\captionsetup[figure]{labelfont=bf, labelsep=space, font=normalsize}
\captionsetup[table]{labelfont=bf, labelsep=space, font=normalsize}
\numberwithin{figure}{chapter}      % 圖按章節編號
\numberwithin{table}{chapter}       % 表按章節編號
\renewcommand{\thefigure}{\arabic{chapter}-\arabic{figure}}  % 圖編號如 1-1, 2-3
\renewcommand{\thetable}{\arabic{chapter}-\arabic{table}}    % 表編號如 1-1, 2-3

% === 程式碼與演算法 ===
\usepackage{listings} % 程式碼顯示(基本)
\usepackage{minted}   % 使用 pygments 的高亮程式碼顯示
\usepackage{fancyvrb} % 進階程式碼排版(Verbatim)
\usepackage{algorithm, algpseudocode} % 演算法顯示
\counterwithin{listing}{chapter}
\renewcommand{\thelisting}{\arabic{chapter}-\arabic{listing}}
%\renewcommand*{\listlistingname}{程式片段目錄}
\setminted{
  style=tango,
  linenos,
  breaklines,
  fontsize=\small
}

% === 參考文獻管理、連結與網址 ===
\usepackage[backend=biber,style=numeric,sorting=none]{biblatex} % 文獻引用
\usepackage{url}      % 顯示網址連結
\usepackage{hyperref} % 超連結相關

%% 加入參考檔案
\addbibresource{references.bib}
\hypersetup{
  colorlinks=true,
  linkcolor=black,      % TOC、section、ref 這些連結黑色
  citecolor=black,      % \cite 用黑色
  filecolor=black,      
  urlcolor=blue,        % 只有網址使用藍色
  pdfborder={0 0 0}     % 不畫紅框、藍框
}

% === 個人常用指令封裝 ===

% \Image[<圖片參數>]{<Label>}{<ImagePath>}{<Caption>}
\newcommand{\Image}[4][width=\linewidth]{
  \begin{figure}[H]
    \centering
    \includegraphics[#1]{figures/#3}
    \caption{#4}
    \label{#2}
  \end{figure}
}

% \Table[4][]{}
\newcommand{\Table}[4][]{
  \begin{table}[H]
    \centering
    \begin{tabular}{#1}
      \toprule
      #3
      \bottomrule
    \end{tabular}
    \caption{#4}
    \label{#2}
  \end{table} 
}

% 標題建立指令 %
\newcommand\zhHeader[1]{
    \begin{center}
    \makebox[4cm][s]{\textbf{\LARGE{#1}}}\\
    \end{center}
    \addcontentsline{toc}{chapter}{#1}
}

\newcommand\enHeader[1]{
    \begin{center}
    \makebox[3cm][c]{\textbf{\LARGE{#1}}}\\
    \end{center}
    \addcontentsline{toc}{chapter}{#1}
}

\defbibheading{bib}{\zhHeader{Bibliography}}

\newcommand{\Endpoint}{\textit{Endpoint}}
\newcommand{\Gateway}{\textit{Gateway}}
\newcommand{\Route}{\textit{Route}}
\newcommand{\Middleware}{\textit{Middleware}}
\newcommand{\Handler}{\textit{Handler}}
\newcommand{\Service}{\textit{Service}}
\newcommand{\Context}{\textit{Context}}
\newcommand{\Capability}{\textit{Capability}}
\newcommand{\Hook}{\textit{Hook}}
\newcommand{\Plugin}{\textit{Plugin}}

\newcommand{\Router}{\textit{Router}}
\newcommand{\Transport}{\textit{Transport}}
\newcommand{\Producer}{\textit{Producer}}
\newcommand{\Consumer}{\textit{Consumer}}
\newcommand{\Capabilities}{\textit{Capabilities}}
\newcommand{\Negotiate}{\textit{Negotiate}}