% 常用套件引入
\usepackage{hyperref}               % 支持超連結和引用
\usepackage{amsmath, amssymb}       % 高級數學符號
\usepackage{booktabs}               % 表格橫線
\usepackage{array}                  % 表格設定
\usepackage{enumitem}               % 列表項目格式
\usepackage{float}                  % 浮動環境控制
\usepackage{listings}               % 程式碼顯示
\usepackage{algorithm, algorithmic} % 演算法顯示
\usepackage{multirow}               % 表格合併行
\usepackage{subcaption}             % 圖片的子圖顯示
\usepackage{xcolor}                 % 顏色設置
\usepackage{graphicx}               % 插入圖片
\usepackage{etoolbox}               % 提供 pretocmd 這類命令
\usepackage{setspace}               % 行距設定
\usepackage{ragged2e}               % 文本對齊控制(如 \justifying)
\usepackage{tabularx}               % 自動調整欄寬的表格
\usepackage{url}                    % 顯示網址連結
\usepackage{ctex}                   % 中文支持

% 浮水印
\usepackage{background}
\backgroundsetup{
  scale=0.5,
  angle=0,
  opacity=0.1,
  contents={\includegraphics{ntust}}
}

% 標題指令
\newcommand\zhHeader[1]{
    \addcontentsline{toc}{chapter}{#1}
    \begin{center}
    \makebox[4cm][s]{\textbf{\LARGE{#1}}}\\
    \end{center}
}

\newcommand\enHeader[1]{
    \addcontentsline{toc}{chapter}{#1}
    \begin{center}
    \makebox[3cm][c]{\textbf{\LARGE{#1}}}\\
    \end{center}
}

% 字體編碼與顏色
\usepackage{fontspec}
\usepackage{xeCJK}
\usepackage[UTF8,heading = true]{ctex}

% 該網址包含 Overleaf 支援的字體
% https://www.overleaf.com/learn/latex/Questions/Which_OTF_or_TTF_fonts_are_supported_via_fontspec
\setmainfont{Times New Roman}
\setCJKmainfont[AutoFakeBold=3]{AR PL UKai TW}
\setCJKmonofont{AR PL UKai TW}
\XeTeXlinebreaklocale "zh" %這兩行一定要加,中文才能自動換行 
\XeTeXlinebreakskip = 0pt plus 1pt
\usepackage{xcolor} % 支持顏色
\color{black}       % 設置字體顏色為黑色

% 版面配置
\usepackage[a4paper, top=3cm, bottom=2cm, left=3cm, right=3cm]{geometry}

% 行距&換行 設定
\usepackage{setspace}
\onehalfspacing % 設置行距為 1.5 倍行高

% 摘要與標題設置
\usepackage{titlesec}
\titleformat{\chapter}[display]
  {\bfseries\fontsize{24pt}{24pt}\selectfont}{\chaptername\ \thechapter}{20pt}{\bfseries\fontsize{20pt}{24pt}\selectfont}
  \titlespacing*{\chapter}{0pt}{-20pt}{40pt} % 題目與標題空間設置

% 字體大小與標題設置
\usepackage{titlesec}
\usepackage{zhnumber} % 阿拉伯數字改成中文數字,英文論文把 zhnumber
\makeatletter
\@addtoreset{section}{chapter}
\@addtoreset{subsection}{section}
\@addtoreset{subsubsection}{subsection}
\@addtoreset{paragraph}{subsubsection}
\pretocmd{\chapter}{\setcounter{section}{0}}{}{}

% Refenence 設定
\usepackage[backend=biber,style=numeric,sorting=none]{biblatex}
\addbibresource{references.bib}
\defbibheading{zhbib}{
  \zhHeader{參考文獻}
}
\defbibheading{enbib}{
  \enHeader{References}
}

% 段落樣式
\newcommand{\UseChineseStyle}{
  % 中文論文
  \renewcommand{\thesection}{\zhnumber{\arabic{section}}}
  \renewcommand{\thesubsection}{\arabic{subsection}}
  \renewcommand{\thesubsubsection}{\arabic{subsubsection}}
  \renewcommand{\theparagraph}{\arabic{paragraph}}

  \titleformat{\chapter}[block] % "章(chapter)"標題 20pt,置中
    {\centering\bfseries\fontsize{20pt}{20pt}\selectfont}
    {第\,\zhnumber{\thechapter}\,章}{20pt}{\centering}
  
  \titleformat{\section}[block] % "節(section)"標題 18pt,置中
    {\raggedright\bfseries\fontsize{18pt}{18pt}\selectfont}
    {(\thesection)}{12pt}{}
  
  \titleformat{\subsection}[block] % "小節(subsection)"標題 16pt,左對齊
    {\raggedright\bfseries\fontsize{16pt}{16pt}\selectfont}
    {\arabic{subsection}.}{12pt}{}
    
  \titlecontents{chapter}[0pt]
  {\bfseries}{第\,\zhnumber{\thecontentslabel}\,章\quad}{}
  {\titlerule*[1pc]{.}\contentspage}

  \titlecontents{section}[2em]
  {}{(\thecontentslabel)\quad}{}
  {\titlerule*[1pc]{.}\contentspage}

  \renewcommand{\listfigurename}{\makebox[4cm][s]{\textbf{\LARGE{圖目錄}}}}
  \renewcommand{\listtablename}{\makebox[4cm][s]{\textbf{\LARGE{表目錄}}}}
  \renewcommand{\contentsname}{\makebox[4cm][s]{\textbf{\LARGE{目錄}}}}
}

\newcommand{\UseEnglishStyle}{
  \renewcommand{\thechapter}{\arabic{chapter}}
  \renewcommand{\thesection}{\thechapter.\arabic{section}}
  \renewcommand{\thesubsection}{\thechapter.\arabic{section}.\arabic{subsection}}
  \renewcommand{\thesubsubsection}{\thechapter.\arabic{section}.\arabic{subsection}.\arabic{subsubsection}}


  \titleformat{\chapter}[block]
    {\centering\bfseries\fontsize{20pt}{20pt}\selectfont}
    {\thechapter.}{20pt}{\centering}

  \titleformat{\section}[block]
    {\raggedright\bfseries\fontsize{18pt}{18pt}\selectfont}
    {\thesection}{12pt}{}

  \titleformat{\subsection}[block]
    {\raggedright\bfseries\fontsize{16pt}{16pt}\selectfont}
    {\thesubsection}{12pt}{}

  \titleformat{\subsubsection}[block]
    {\raggedright\bfseries\fontsize{14pt}{14pt}\selectfont}
    {\thesubsubsection}{12pt}{}

  \renewcommand{\listfigurename}{List of Figures}
  \renewcommand{\listtablename}{List of Tables}
  \renewcommand{\contentsname}{Contents}
}

% 頁碼設置
\usepackage{fancyhdr}             % 頁眉頁腳設置
\pagestyle{fancy}
\fancyhf{}
\fancyfoot[C]{\thepage}           % 頁碼置於下方中間

% 圖表標題與編號設置
\usepackage{caption}
\captionsetup[figure]{labelfont=bf, labelsep=period, font=normalsize}
\captionsetup[table]{labelfont=bf, labelsep=period, font=normalsize}
\renewcommand{\thefigure}{\thechapter-\arabic{figure}}  % 圖編號如 1-1, 2-3
\renewcommand{\thetable}{\thechapter-\arabic{table}}    % 表編號如 1-1, 2-3
\numberwithin{figure}{chapter}                          % 圖按章節編號
\numberwithin{table}{chapter}                           % 表按章節編號
\renewcommand{\figurename}{圖}
\renewcommand{\tablename}{表}

\usepackage{titletoc}
\titlecontents{tableofcontents}
  [0pt]                              % 縮排
  {\bfseries}                        % above code
  {\contentslabel{2em}}              % numbered entry format
  {}                                 % numberless entry format
  {\titlerule*[1pc]{.}\contentspage} % filler and page number format


% 自訂圖片插入指令(固定位置、置中、標題)
\newcommand{\InsertFigure}[3][]{%
  \begin{figure}[H]
    \centering
    \includegraphics[width=0.8\linewidth]{#2}
    \caption{#3}
    \ifx&#1&%
    \else
      \label{#1}
    \fi
  \end{figure}
}

% \InsertFigure{}{圖片路徑}{說明}
% \InsertFigure[關聯]{圖片路徑}{說明}

% 自訂表格插入指令(固定位置、置中、標題)
\newcommand{\InsertTable}[4][]{%
  \begin{table}[H]
    \centering
    \begin{tabular}{#2}
      \toprule
      #3
      \bottomrule
    \end{tabular}
    \caption{#4}
    \ifx\\#1\\%
    \else
      \label{#1}
    \fi
  \end{table}
}

