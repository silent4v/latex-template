\chapter{插入程式碼片段與演算法}
\label{chap4}

在技術與學術文件中,清晰展示程式碼與演算法邏輯是極為重要的排版需求。LaTeX 提供了多種高品質工具來實現此目標,其中以 \texttt{minted} 以及 \texttt{algorithm} 搭配 \texttt{algpseudocode} 最為常用。本章將說明兩者的使用方式與適用情境。

\section{使用 \texttt{minted} 插入語法高亮程式碼}

\texttt{minted} 套件利用 \texttt{Pygments} 提供跨語言的語法高亮,能準確保留縮排與語意格式,適合展示原始碼片段。

\subsection*{使用前置設定}

使用 \texttt{minted} 時,請於前言區引入套件:

\begin{verbatim}
\usepackage{minted}
\usepackage{xcolor}
\end{verbatim}    

並在編譯時使用: \texttt{pdflatex -shell-escape main.tex}

\subsection*{範例語法}

\begin{verbatim}
\begin{listing}[H]
  \begin{minted}[<樣式設定>]{<選擇語言>}
  <程式碼內容>
  \end{minted}
  \caption{<說明>}
  \label{<標籤>}
\end{listing}
\end{verbatim}

以下為一段 C++ 程式碼片段的範例:
\begin{listing}[H]
  \begin{minted}[fontsize=\small, linenos,style=vs, frame=lines, bgcolor=gray!1]{cpp}
  #include <bits/stdc++.h>
  using namespace std;
  int binarySearch(int arr[], int low, int high, int x)
  {
      while (low <= high) {
          int mid = low + (high - low) / 2;
  
          // Check if x is present at mid
          if (arr[mid] == x)
              return mid;
  
          // If x greater, ignore left half
          if (arr[mid] < x)
              low = mid + 1;
  
          // If x is smaller, ignore right half
          else
              high = mid - 1;
      }
  
      // If we reach here, then element was not present
      return -1;
  }
  \end{minted}
  \caption{二分搜尋法}
  \label{algo:BinarySearch}
\end{listing}
可選參數說明如下:

\begin{itemize} 
\item \texttt{style}:選擇風格,這裡使用 vs 風格
\item \texttt{linenos}:顯示行號
\item \texttt{frame=lines}:上下加框線
\item \texttt{bgcolor=gray!1}:背景顏色
\item \texttt{fontsize=\small}:字體大小
\end{itemize}
\section{使用 \texttt{algorithm} 插入演算法}

若需表達邏輯流程、條件分支與步驟運算,建議使用 \texttt{algorithm} 套件搭配 \texttt{algpseudocode},以排版接近 pseudo-code 形式展示演算法。

\subsection*{前置設定}

請在 preamble 加入以下指令:

\begin{verbatim}
\usepackage{algorithm}
\usepackage{algpseudocode}
\end{verbatim}

\subsection*{基本範例}

\begin{verbatim}
\begin{algorithm}
    \caption{<說明>}
    \label{<標籤>}
    \begin{algorithmic}[1]
    <算法描述>
    \end{algorithmic}
\end{algorithm}
\end{verbatim}

以下為一個遞迴階乘演算法的範例:

\begin{algorithm}
\caption{階乘演算法}
\label{alg:factorial}
\begin{algorithmic}[1]
    \Function{Factorial}{n}
        \If{$n = 0$}
            \State \Return $1$
        \Else
            \State \Return $n \times \Call{Factorial}{n - 1}$
        \EndIf
    \EndFunction
\end{algorithmic}
\end{algorithm}

\subsection*{語法元素說明}

\begin{itemize}
\item \texttt{Function, If, Return, Call} 為結構性語句
\item \texttt{\textbackslash State} 用於表示單行操作
\item \texttt{[1]} 可為演算法加上行號
\end{itemize}

\section{關於 \texttt{algorithm} 的 \texttt{\textbackslash label} 放置位置}

在使用 \texttt{algorithm} 環境搭配 \texttt{caption} 產生標題與標號時,標籤(\verb|\label|)的放置位置會影響引用的正確性。這與一般 \texttt{figure} 或 \texttt{table} 環境略有不同。

\texttt{\textbackslash label} 必須放在 \texttt{\textbackslash begin\{algorithm\}} 之後,且*緊接著放在 \texttt{\textbackslash caption} 之後才有效產生正確的標號引用*。只有 \texttt{\textbackslash begin\{algorithm\}} 在 begin 後直接設定 caption。figure, table, listing 都在最後的 \texttt{\textbackslash begin\{...\}} 前在設定 caption 與 label 就好。

\begin{verbatim}
\begin{algorithm}
\caption{演算法名稱}
\label{alg:example} % ← 要放在 caption 之後
\begin{algorithmic}
  ...
\end{algorithmic}
\end{algorithm}
\end{verbatim}

\paragraph{常見錯誤:}

若將 \verb|\label| 放在 \verb|\begin{algorithm}| 之前,或放在 \verb|\begin{algorithmic}| 之後,將導致 \verb|\ref{alg:example}| 引用失效或標號錯誤。

\paragraph{技術說明:}

這是因為 LaTeX 中的浮動體(如 figure、table、algorithm)之編號是在 \verb|\caption| 處才產生的。若 \verb|\label| 放在 caption 之前,當時該浮動體尚未取得編號,導致標號參照錯誤。

\paragraph{建議習慣:}

為確保一致性與可維護性,建議一律將 \verb|\label| 緊接著放在 \verb|\caption| 後,如下所示:

\begin{verbatim}
\begin{algorithm}
\caption{XXX 方法的流程}
\label{alg:xxx}
\begin{algorithmic}[1]
  \State ...
\end{algorithmic}
\end{algorithm}
\end{verbatim}

如此能保證使用 \verb|\ref{alg:xxx}| 或 \verb|\autoref{alg:xxx}| 引用時正確顯示對應編號。
