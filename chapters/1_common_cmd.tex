\chapter{編譯設定與常用指令}
\label{chap1}

編譯是一個常見的問題,這裡簡單描述一下:\texttt{\textbackslash input} 行為大概就是直接複製貼上那個 \texttt{.tex} 的內容,\texttt{\textbackslash include}比較像是鏈結進來,像是 C 的編譯器把 \texttt{"a.o"} 跟 \texttt{"b.o"} 合併起來的感覺。

\section{基本差異比較}

\begin{itemize}
  \item \textbf{\texttt{\textbackslash include}}:用於章節級別的插入,會在插入前後自動加入換頁,適合用於引入整章內容(如第 1 章、第 2 章)。
  \item \textbf{\texttt{\textbackslash input}}:用於片段內容的插入,如一段說明、一份表格或一段自訂命令,不會換頁。
\end{itemize}

\section{使用範例}

假設我們有以下檔案結構:

\begin{verbatim}
main.tex
chapters/
├── chapter1.tex
├── chapter2.tex
└── sections/def.tex
\end{verbatim}

在 \texttt{main.tex} 中,我們這樣插入章節:

\begin{verbatim}
\include{chapters/chapter1}
\include{chapters/chapter2}
\end{verbatim}

\texttt{chapter1.tex} 中,也可以插入子段落:\texttt{\textbackslash input\{sections/def.tex\}}

\section{注意事項}

\begin{itemize}
  \item \texttt{\textbackslash include} 不能巢狀使用。你不能在 \texttt{chapter1.tex} 裡再寫 \texttt{\textbackslash include}。
  \item \texttt{\textbackslash include} 支援 \texttt{\textbackslash includeonly} 指令,可以只編譯特定章節,加快草稿時的速度。(但是要放在 \texttt{\textbackslash begin\{document\}} Document 之前)
  \item 檔案名稱不需要加上 \texttt{.tex} 副檔名。
  \item 無論是 \texttt{\textbackslash include} 或 \texttt{\textbackslash input},都必須寫在 \texttt{\textbackslash begin\{document\}} 之後。
\end{itemize}

\section{建議使用方式}

\begin{itemize}
  \item 使用 \texttt{\textbackslash include} 管理各章節(例如 \texttt{Chapter 1, 2, 3})。
  \item 在章節中,使用 \texttt{\textbackslash input} 引入表格、插圖說明、定義區塊、命令模板等。
\end{itemize}

\section{newcommand}

最後解釋一下 \texttt{\textbackslash newcommand} 的用途,通常用法是:

\begin{verbatim}
\newcommand{ <name> }[]{ <defintion> }
\newcommand{ <name> }{ <defintion> }
\end{verbatim}

在\LaTeX 中,\texttt{[]} 跟 \texttt{\{\}} 的差別是 \texttt{[]} 是可選的參數,\texttt{\{\}} 則為必要參數。

以本文件來說,\texttt{main.tex} 最上方的:


\begin{verbatim}
\ documentclass[a4paper,12pt,fontset=none]{report}
\end{verbatim}

\texttt{[a4paper,12pt,fontset=none]}是完全可以省略的。\texttt{\textbackslash newcommand}行為大概是定義一個 Function,\texttt{frontmatter.tex}最上方就先定義好中文摘要、Abstract、致謝顯示的標題,並且在\texttt{acknowledge.tex} 有以下語法:

\begin{verbatim}
\chapter*{\enAbstractTitle}
\addcontentsline{toc}{chapter}{\enAbstractTitle}
\normalsize
\end{verbatim}

其行為就是:我想要宣告一個章節,並且這個章節不要編號,並且加入到目錄中(toc = table of content, 就是指目錄)。然後使用 \texttt{\textbackslash enAbstractTitle}把前面定義好的章節標題引入進來。