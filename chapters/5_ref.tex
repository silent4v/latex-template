\chapter{交叉引用與文獻引用機制}
\label{chap5}

LaTeX 提供強大且嚴謹的交叉引用機制,可讓使用者在長篇文件中準確引用圖表、章節、演算法與外部文獻,維持內容一致性並提升可讀性。本章將介紹四種常見引用指令:\verb|\ref|、\verb|\autoref|、\verb|\nameref| 與 \verb|\cite|,並說明其使用情境與差異。除了\verb|\cite|以外,這些都要跟 \verb|\label| 語法進行搭配。

\section{標號引用:\texttt{\textbackslash ref}}

\verb|\ref| 是最基本的標號引用指令,僅輸出對應編號本身,不包含任何類型前綴。使用者需手動加上詞彙(如「圖」、「表」、「演算法」):

\begin{verbatim}
如圖~\ref{fig:architecture} 所示 ...
\end{verbatim}

輸出結果範例:

> 如圖~3 所示 ...

此語法適用於細緻排版控制,但需使用者自行維持用詞一致。

\section{自動標號類型:\texttt{\textbackslash autoref}}

\verb|\autoref| 來自 \texttt{hyperref} 套件,會根據標籤所屬環境自動加上類型前綴(如 Figure、Table、Algorithm 等),語法簡潔且一致性高:

\begin{verbatim}
\autoref{alg:factorial}
\end{verbatim}

輸出結果:

> Algorithm 1

這對於長篇文件尤為實用,能大幅減少人工錯誤,但前綴會依語言預設值(英文),若需中文可使用 \texttt{cleveref} 套件替代。

\section{引用名稱:\texttt{\textbackslash nameref}}

\verb|\nameref|(需載入 \texttt{nameref} 套件,或搭配 \texttt{hyperref})用來輸出標籤對應的標題文字,非常適合自然語言式敘述:

\begin{verbatim}
詳見「\nameref{sec:evaluation} 」一節。
\end{verbatim}

輸出結果:

> 詳見「效能評估」一節。

此語法對於強調章節標題內容而非編號的敘述情境特別有用,可與 \verb|\ref| 或 \verb|\autoref| 搭配使用。

\section{文獻引用:\texttt{\textbackslash cite}}

\verb|\cite| 用於插入文獻來源標號,通常配合 BibTeX 或 BibLaTeX 資料庫管理文獻。引用格式與排序會由所選格式決定,例如:

\begin{verbatim}
如文獻~\cite{webrtc-overview} 所述 ...
\end{verbatim}

\begin{itemize}
\item 如RFC3350~\cite{rfc3550} 所述 ...
\item 參考 Gstreamer~\cite{gstreamer-gitlab}網站 ...
\item 根據 SRT~\cite{srt-protocol} 協定 ...
\item 此篇論文 ~\cite{hoang2018dynamic} 指出...
\end{itemize}

這樣就可以引用參考資料了。在本模板中,已經使用 \verb|\addbibresource{references.bib}| 把 references.bib 加入參考來源了,只需要把對應的引用格式丟入該檔案,並確保 main.tex 的最下方有使用 \verb|\printbibliography[heading=zhbib]| 渲染參考文獻即可