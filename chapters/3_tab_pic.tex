\chapter{圖片與表格語法}
\label{chap3}

在技術寫作與學術排版中,圖像與表格扮演著不可或缺的角色。Markdown 提供簡單直覺的語法以快速建立圖片與表格,而 LaTeX 則透過強大的浮動物件機制與格式控制,能夠呈現更高品質且可引用的版面效果。以下將介紹 Markdown 與 LaTeX 在圖片與表格處理上的語法差異與對應方法。

\section{排版邏輯}

在 LaTeX 中,圖與表格通常被視為「浮動體(float objects)」,也就是說它們不一定會出現在程式碼撰寫的確切位置,而是由排版引擎根據整體頁面美觀與空間安排自動決定擺放處。為此,LaTeX 提供一組定位參數,來「建議」浮動體的擺放位置:

\begin{itemize}
  \item \texttt{h}:放在此處(here)
  \item \texttt{t}:放在頁面頂部(top)
  \item \texttt{b}:放在頁面底部(bottom)
  \item \texttt{p}:放在單獨一頁(page of floats)
\end{itemize}

這些選項可組合使用,例如 \texttt{[htbp]} 表示「優先嘗試此處放置,否則嘗試頁首、頁尾或浮動頁」。

然而,若圖片或表格位置附近空間不足、頁面排版邏輯無法滿足條件,LaTeX 將自動「推遲」浮動體的呈現,導致圖表延後出現,有時甚至出現在相當遙遠的段落或頁面,造成閱讀混亂。

\paragraph{強制位置:\texttt{[H]} 的使用}

若希望圖表**完全出現在原地**,不讓 LaTeX 自動移動,需使用 \texttt{float} 套件,並改用 \texttt{[H]} 選項(注意是大寫 H):

\begin{verbatim}
\usepackage{float} % 加在 preamble

\begin{figure}[H]
  \centering
  \includegraphics[width=0.8\textwidth]{example.png}
  \caption{強制定位的圖片}
  \label{fig:fixed}
\end{figure}
\end{verbatim}

此寫法會完全抑制浮動特性,強制圖片出現在當前位置,適合用於教學文章、逐步範例說明、或需圖文密切配合之場合。

\paragraph{使用 \texttt{[H]} 的注意事項}

儘管 \texttt{[H]} 提供精準控制,但過度使用可能導致頁面佈局凌亂,因為 LaTeX 排版引擎會喪失自動美化整體頁面的能力。因此,建議僅在下列情況使用:

\begin{itemize}
  \item 教學文檔中,圖表需緊貼程式碼或說明。
  \item 小型圖片或表格不影響頁面排版。
  \item 實驗結果、逐步操作等需逐段插圖說明的場合。
\end{itemize}

綜合而言,\texttt{[htbp]} 提供彈性排版,而 \texttt{[H]} 則提供嚴格控制。依據文件性質與可接受的排版自由度選擇合適方式,將有助於產出既專業又可讀性良好的文檔。

\section{插入圖片}

Markdown 的圖片語法採用驚嘆號與中括號及小括號結合,語法如下:

\begin{verbatim}
![替代文字](圖片網址或路徑)
\end{verbatim}

在 LaTeX 中,插圖須透過 \texttt{graphicx} 套件輔助,建議使用 \texttt{figure} 浮動環境,範例如下:

\begin{verbatim}
\usepackage{graphicx} % 於環境加入
\begin{figure}[htbp]
  \centering
  \includegraphics[width=0.8\textwidth]{ntust.png}
  \caption{範例圖片說明}
  \label{fig:example}
\end{figure}
\end{verbatim}

\Image[width=0.2\textwidth]{fig:example}{ntust.png}{範例圖片說明}


上述語法說明如下:

\begin{itemize}
  \item \texttt{figure} 為浮動環境,控制圖像位置與標號。
  \item \texttt{width=0.8\textbackslash textwidth} 控制寬度占比,可改用絕對尺寸。
  \item \texttt{caption} 提供圖說,會自動產生圖表編號。
  \item \texttt{label} 可搭配 \texttt{ref} 或 \texttt{autoref} 進行交叉引用。
\end{itemize}

建議圖片格式使用 PDF、EPS 等向量圖,若為點陣圖則解析度應高於 300 DPI,以確保印刷品質。

\section{製作表格}

Markdown 的表格語法簡潔,但僅支援基本資料排列與對齊:

\begin{verbatim}
| 名稱 | 數量 |
|------|------|
| 筆電 |  1   |
| 滑鼠 |  2   |
\end{verbatim}

在 LaTeX 中,表格需透過 \texttt{tabular} 或 \texttt{table} 浮動環境撰寫。基本範例如下:

\begin{verbatim}
\begin{table}[htbp]
  \centering
  \begin{tabular}{|c|c|}
    \hline
    名稱 & 數量 \\
    \hline
    筆電 & 1 \\
    滑鼠 & 2 \\
    \hline
  \end{tabular}
  \caption{設備清單表}
  \label{tab:devices}
\end{table}
\end{verbatim}

\begin{table}[htbp]
  \centering
  \begin{tabular}{|c|c|}
    \hline
    名稱 & 數量 \\
    \hline
    筆電 & 1 \\
    滑鼠 & 2 \\
    \hline
  \end{tabular}
  \caption{設備清單表}
  \label{tab:devices}
\end{table}

欄位格式代碼說明如下:

\begin{itemize}
  \item \texttt{l}:靠左對齊(left)
  \item \texttt{c}:置中對齊(center)
  \item \texttt{r}:靠右對齊(right)
  \item 加上 \texttt{|} 表示加上垂直分隔線
\end{itemize}

如需進一步美化表格,推薦載入 \texttt{booktabs} 套件,使用 \texttt{\textbackslash toprule}、\texttt{\textbackslash midrule}、\texttt{\textbackslash bottomrule} 等專業排版指令取代 \texttt{\textbackslash hline}。

\paragraph{延伸功能}

若遇表格過寬或需要跨頁顯示,可結合以下套件進行擴展:

\begin{itemize}
  \item \texttt{tabularx}:支援自動欄寬調整。
  \item \texttt{adjustbox}:可將表格縮放或旋轉。
  \item \texttt{longtable}:支援跨頁表格排版。
\end{itemize}


\begin{table}[H]
  \centering
  \begin{tabular}{lcr}
    \toprule
    設備名稱 & 數量 & 備註 \\
    \midrule
    筆電     & 1   & 公司配發 \\
    滑鼠     & 2   & 自購 \\
    鍵盤     & 1   & 自購 \\
    \bottomrule
  \end{tabular}
  \caption{美化過的表格}
  \label{tab:equipment}
\end{table}

\section{圖片與表格尺寸控制對照表}

為提高排版彈性,LaTeX 提供多種尺寸單位與擴充指令供圖表配置使用。下表彙整常見語法與其用途:

\begin{table}[H]
  \centering
  \begin{tabular}{@{}llp{7cm}@{}}
    \toprule
    \textbf{語法} & \textbf{單位} & \textbf{說明與用途} \\
    \midrule
    \verb|\textwidth| & 寬度 & 整體文字區域的寬度,通常用於全文圖表寬度統一 \\
    \verb|\linewidth| & 寬度 & 當前區塊(如 minipage、list 環境)中的可用寬度 \\
    \verb|width=...|   & 寬度 & 設定圖像或表格的最大寬度,單位可為 \verb|\textwidth|、\verb|\linewidth|、cm 等 \\
    \verb|height=...|  & 高度 & 設定圖像或表格的最大高度,常與 \verb|keepaspectratio| 搭配使用以避免圖像變形 \\
    \verb|keepaspectratio| & 旗標 & 用於圖像縮放時保持寬高比例,避免變形 \\
    \verb|\arraystretch| & 倍數 & 用於拉高表格的行距,提升可讀性(預設值為 1.0) \\
    \verb|adjustbox| 套件 & 縮放控制 & 可同時調整圖表的寬與高,並支援置中、旋轉等進階功能 \\
    \bottomrule
  \end{tabular}
  \caption{LaTeX 中圖表尺寸控制常用語法總覽}
  \label{tab:dimension-control}
\end{table}

下面筆者提供一個簡單的 \texttt{\textbackslash Image}指令,分別使用 \texttt{label,imagePath,caption}當作參數插入圖片。
label的命名,個人習慣使用 \texttt{fig:} 當圖片的前綴;\texttt{tab:} 當表格的前綴。\texttt{\textbackslash label} 是一個很重要的語法,他可以建立一個可參考的位置。

\begin{verbatim}
\usepackage{graphicx}
\usepackage{caption}
\usepackage{float} % 若使用 [H]

% 定義 \Image 指令
\newcommand{\Image}[4][]{%
  \begin{figure}[H]
    \centering
    \includegraphics[#1]{#3}
    \caption{#4}
    \label{fig:#2}
  \end{figure}
}
\end{verbatim}