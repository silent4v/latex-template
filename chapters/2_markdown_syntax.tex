\chapter{Markdown 語法映射}
\label{chap2}

在撰寫技術文件時,Markdown 與 LaTeX 是兩種常見的標記語言。Markdown 語法簡潔直觀,適用於快速記錄與網頁排版,而 LaTeX 則因其排版精細與學術用途而被廣泛使用於論文撰寫。下列將介紹 Markdown 中常見的語法元素,並說明其在 LaTeX 中的對應寫法。

\section{標題結構對應}

Markdown 中以井字號(\#)表示標題階層,從 一個 \# 到 六個 \# 分別對應第一至第六層標題。LaTeX 中則使用章節結構指令來表達對應層級,如下所示:

\begin{center}
\begin{tabular}{ll}
\toprule
\textbf{Markdown 標題語法} & \textbf{對應 LaTeX 指令} \\
\midrule
\# Header 1        & \verb|\chapter{...}|         \\
\#\# Header 2      & \verb|\section{...}|         \\
\#\#\# Header 3    & \verb|\subsection{...}|      \\
\#\#\#\# Header 4  & \verb|\subsubsection{...}|   \\
\#\#\#\#\# Header 5& \verb|\paragraph{...}|       \\
\#\#\#\#\#\# Header 6 & \verb|\subparagraph{...}| \\
\bottomrule
\end{tabular}
\end{center}

\section{清單結構}

Markdown 支援有序與無序清單,分別使用數字與破折號(-)表示。在 LaTeX 中,對應為 \verb|enumerate| 與 \verb|itemize| 環境:

有序清單(Ordered List)
\begin{verbatim}
\begin{enumerate}
  \item 第一項
  \item 第二項
  \item 第三項
\end{enumerate}
\end{verbatim}

\begin{enumerate}
  \item 第一項
  \item 第二項
  \item 第三項
\end{enumerate}

無序清單(Unordered List)
\begin{verbatim}
\begin{itemize}
  \item 第一項
  \item 第二項
  \item 第三項
\end{itemize}
\end{verbatim}

\begin{itemize}
  \item 第一項
  \item 第二項
  \item 第三項
\end{itemize}

\section{引用區塊(Blockquote)}

Markdown 使用 \texttt{>} 表示引用。在 LaTeX 中,可使用 \verb|quote| 環境:

\begin{verbatim}
\begin{quote}
這是一段引用文字。
\end{quote}
\end{verbatim}

\begin{quote}
這是一段引用文字。
\end{quote}

\section{文字樣式}

Markdown 提供簡易的文字樣式控制,以下為其在 LaTeX 中的對應方式:

\begin{itemize}
  \item 粗體:LaTeX 使用 \verb|\textbf{Text}|
  \item 斜體:LaTeX 使用 \verb|\textit{Text}|
  \item 底線:LaTeX 使用 \verb|\uline{Text}|(需載入 \verb|ulem| 套件)
\end{itemize}

\section{程式碼表示法(Code)}

\paragraph{行內程式碼}
\begin{itemize}
  \item Markdown:\verb|`code`|
  \item LaTeX:\verb|\texttt{code}|
\end{itemize}

\paragraph{區塊程式碼}
基本用法是 \verb|listings| 套件輔助顯示:

\begin{verbatim}
\usepackage{listings}
...
\begin{lstlisting}[language=Python]
def hello():
    print("Hello, World!")
\end{lstlisting}
\end{verbatim}

% 但是這套件的效果其實滿差的,筆者會建議使用 minted(~\nameref{chap4} 會介紹),在 Overleaf 上有直接整合。透過上述對應說明,使用者可將一些基本 Markdown 編寫習慣轉換為 LaTeX 撰寫。

% 而對於表格、圖片,將會在下一章「~\nameref{chap3}」說明,因為 Table 與 Picture 在 LaTeX 算是一個比較複雜的主題
