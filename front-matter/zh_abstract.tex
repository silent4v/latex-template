\newpage
\zhHeader{\zhAbstractTitle}
\normalsize
%%%%

本文件依據國立臺灣科技大學論文格式建議撰寫,作為撰寫學術文件時的起始範本。考量到現代多數使用者已習慣以 Markdown 撰寫筆記與報告,本模板設計的核心目標之一,是協助使用者從 Markdown 思維順利過渡至 \LaTeX{} 的語法結構。

本文各章節將逐步介紹 Markdown 常見語法與其在 \LaTeX{} 中的對應方式,並說明如何在排版過程中加入演算法(pseudo code)、表格(table)、圖片(figure)等常見結構,進一步引導讀者理解如交叉參照(\verb|\ref|、\verb|\nameref|)與文獻引用(\verb|\cite|)等功能的使用方式與排版效果。

為了協助讀者快速理解整體架構,模板亦於下方列出各主要組成檔案的功能說明:

\begin{itemize}
  \item \textbf{environment.tex}:預先引入常用套件並設定整體排版風格。若需調整字型、行距或語言模式,可於此進行全域設定。請在 \texttt{main.tex} 中,於其他 \texttt{tex} 檔案引用之前先行載入。
  
  \item \textbf{\texttt{frontmatter.tex}}:管理論文前言部分,包括摘要、致謝、目錄、圖表清單等內容。各段落內容分別寫於 \texttt{front-matter/} 資料夾中,方便獨立撰寫與維護。
  
  \item \textbf{\texttt{main.tex}}:整體論文的編譯進入點,主要引入各章節(位於 \texttt{chapters/})內容。在撰寫階段,若僅需編譯個別章節(例如 Chapter 2),建議將其他 \texttt{\textbackslash include} 指令暫時註解,能有效縮短編譯時間。撰寫章節時亦可先略過 \texttt{frontmatter} 區段。
\end{itemize}

本模板不僅提供一份可立即套用的排版起點,也試圖在說明與範例中,協助使用者逐步熟悉 \LaTeX{} 的語意標記邏輯,為後續學術文件的撰寫奠定基礎。
