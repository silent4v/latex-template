\newpage
\enHeader{ABSTRACT}
\normalsize

%%%%

This research presents an in-depth analysis and participatory study on the maintenance and contribution process of FFmpeg, a widely adopted open-source multimedia framework. The study explores the system architecture, development workflow, and module evolution from the perspective of a contributor, aiming to examine how large-scale open-source systems operate in practice and how their modular designs affect maintainability and extensibility.

Chapter One introduces the research motivation and background, emphasizing the growing importance of multimedia systems such as streaming, transcoding, and media analysis. FFmpeg, while powerful, poses a steep learning curve for contributors due to its codebase scale and tight module coupling. The chapter defines the scope and goals of this thesis, positioning it as an experience-based exploration of open-source participation.

Chapter Two surveys related systems, including GStreamer, VLC, and Libav, and compares them across modularity, codec support, performance, and community governance. This comparative study establishes FFmpeg’s position in the landscape of multimedia toolkits and provides theoretical context for later analysis.

Chapter Three describes the research methodology, covering environment setup, patch submission, dependency analysis, regression testing, and code quality assessment. The study employs Git history analysis, clang static checks, and Valgrind profiling to ensure systematic and reproducible results.

Chapter Four presents the experimental outcomes, including real-world bug reports, patch contributions, and performance regression cases. Each instance is discussed with respect to its technical context and community feedback, demonstrating the internal dynamics of FFmpeg’s development ecosystem.

Chapter Five concludes with key observations about FFmpeg’s architectural strengths and maintenance challenges. Recommendations are proposed for modular redesign, documentation improvements, and CI enhancements. Future work includes comparative analysis with other large open-source projects to establish a generalized participation model for system-level maintenance.

Keywords: Media Streaming, Audio and Video Processing