\newpage
\zhHeader{中文摘要}
\normalsize

%%%%

本研究以參與開源多媒體處理框架 FFmpeg 專案之實際經驗為基礎,探討其系統架構、維運機制與模組發展歷程,並針對其在現代影音應用中的角色進行技術層面的深入分析。研究過程涵蓋原始碼閱讀、修補提交、社群互動與模組追蹤,透過參與式實作驗證系統維運中的挑戰與模式,並提出相關改善建議。

第一章「緒論」主要說明本研究之背景與動機,指出隨著串流媒體、錄製轉檔與多媒體分析等應用不斷擴增,開源多媒體工具在實務系統中所扮演之關鍵性角色日益明顯。FFmpeg 作為其中最廣泛使用的系統,其龐大的原始碼基礎與模組結構,對於新進貢獻者而言,存在相當高的理解門檻與維運挑戰。本章亦界定本研究之範圍、目標與方法,明確以「從貢獻者視角出發的系統參與」為研究主軸。

第二章「相關研究」回顧開源多媒體處理框架的發展脈絡與主要系統,包括 GStreamer、VLC 及 Libav,並從模組化設計、編解碼支援度、效能與社群治理等層面加以比較。透過文獻與官方文件的彙整,本章建立 FFmpeg 在功能定位與技術策略上的相對座標,亦說明 FFmpeg 如何在高效能與廣泛支援間取得平衡,為後續分析提供理論基礎。

第三章「研究方法」聚焦於本研究所採用之參與策略與技術手段,具體包括建置本地開發環境、熟悉提交與審核流程、進行模組相依分析、回歸測試、以及維運文件撰寫等。研究亦運用 Git 版本控制系統追蹤模組演化,並結合 clang 與 Valgrind 等工具進行錯誤檢測與效能分析。本章強調方法上的系統性與可重現性,以避免主觀經驗偏差。

第四章「實驗結果」呈現本研究在維運過程中所處理之問題類型、提出之修補內容與其影響範圍,包括封包轉封裝錯誤、邊界條件處理失敗、格式自動推斷異常等實例。並透過實驗記錄說明維運過程中的決策依據與與社群互動之反饋效益,展示 FFmpeg 作為大型開源專案其實務運作機制之具體樣貌。

 第五章「結論與未來工作」統整本研究成果,指出 FFmpeg 優異之模組整合能力與編解碼覆蓋率,亦伴隨著高耦合、歷史包袱與進入門檻等維運挑戰。研究建議未來可針對模組邊界重新劃分、增進文件結構與導入 CI 改進進行研究與貢獻,並可進一步延伸至其他開源大型專案之比較分析,以建立通用之開源系統維運參與模型。

關鍵字:媒體串流、影音處理、編解碼器、多媒體技術