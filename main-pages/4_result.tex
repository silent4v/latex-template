\chapter{實驗結果}

本章展示系統於模擬環境中運行之初步結果,以表格與統計方式呈現基本效能指標,供範例參考使用。

\section{實驗環境}

模擬環境設定如下:

\begin{itemize}
    \item 作業系統:Ubuntu 22.04 LTS
    \item 撰寫語言:Node.js v20
    \item 測試模組:FFmpeg 6.0 + mediasoup 3.x
    \item 串流格式:WebRTC + MP4 封裝輸出
\end{itemize}

\section{結果呈現}

本節以表格形式顯示在三組不同模擬負載下之封包遺失率與平均延遲:

\InsertTable[tab:exp]{ccc}{
  \toprule
  測試組 & 平均延遲(ms) & 封包遺失率(\%) \\
  \midrule
  測試組 A & 48 & 0.4 \\
  測試組 B & 71 & 1.1 \\
  測試組 C & 129 & 3.2 \\
  \bottomrule
}{模擬環境下串流效能測試}

\section{結果分析}

由表~\ref{tab:exp} 可觀察,在模擬高負載情境下,封包遺失與延遲量皆有上升趨勢。實作系統未額外實施擴充快取或動態負載均衡策略,故在 C 組實驗中出現明顯瓶頸。

本章所呈資料為範例模擬結果,未實際反映真實系統效能,請使用者依照需求自行測試並更換資料。
