\chapter{緒論}

本章說明本論文所關注之主題背景、研究動機與目的,並介紹論文架構。範例內容以模擬性質撰寫,供後續論文撰寫者參考。

\section{研究背景與動機}

隨著資訊科技的發展,資料處理與系統設計日趨多元,學生在學術或實作研究中,往往需要具備清晰的撰寫結構與論證能力。為協助後續學生建立正確撰寫習慣,本論文提供一份符合研究所格式之 \LaTeX~撰寫範本。

\section{研究目的}

本論文範本旨在展示完整論文撰寫過程之結構範例,包含標題樣式設定、圖表插入、引用格式與段落撰寫方式。使用者可依照實際研究主題,將內容進行替換,快速完成具備學術格式之論文初稿。

\section{研究方法概述}

為呈現完整論文內容結構,範本內包含以下元素:

\begin{itemize}
  \item 多章節分頁與獨立檔案管理(如 \texttt{main-pages/} 路徑)
  \item 插圖與表格(如下圖與下表所示)
  \item 引用參考文獻(使用 \texttt{biblatex} 與 \texttt{biber})
\end{itemize}

下方範例展示圖片插入格式:

\InsertFigure[fig:template-arch]{figures/template-arch.png}{本論文範本之結構示意圖}

圖~\ref{fig:template-arch} 為範本整體章節與檔案組織之視覺化示意,供後續使用者參考修改。

以下展示一個簡單表格範例,說明圖表樣式與排版設定:

\InsertTable[tab:sample]{ccc}{
  \toprule
  編號 & 名稱 & 結果 \\
  \midrule
  1 & 測試一 & 通過 \\
  2 & 測試二 & 失敗 \\
  \bottomrule
}{格式元素對照表}

\InsertTable[tab:sample2]{ccc}{
  \toprule
  項目 & 說明 & 範例 \\
  \midrule
  圖插入 & 使用 \texttt{\textbackslash InsertFigure} & 圖~\ref{fig:template-arch} \\
  表插入 & 使用 \texttt{\textbackslash InsertTable} & 表 \\
  文獻引用 & 使用 \texttt{\textbackslash cite\char`\{\char`\}} \\
  \bottomrule
}{格式元素對照表}

\section{論文架構}

本論文共分為五章,章節內容與重點如下:

\begin{itemize}
  \item 第一章:緒論,說明研究動機、目的與論文組織。
  \item 第二章:相關研究,回顧與主題有關之技術與文獻。
  \item 第三章:研究方法,描述實作內容、使用工具與設計架構。
  \item 第四章:實驗結果,展示實作成果與測試結果。
  \item 第五章:結論與未來工作,總結本論文並提出延伸方向。
\end{itemize}

本章內容為論文開端,應以清晰的邏輯呈現研究的「由來」與「脈絡」,為後續章節奠定基礎。
