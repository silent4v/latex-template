\chapter{研究方法}

本章節說明本範本系統所採用之設計理念與架構規劃,並輔以簡化流程圖與模組功能描述,幫助讀者了解各元件間之互動方式與系統組成。

\section{系統設計架構}

本系統依據典型多媒體伺服架構進行模組劃分,主要包括接收端、編碼處理模組、傳輸模組與授權驗證模組。圖~\ref{fig:flow} 顯示整體資料流程設計。

\InsertFigure[fig:flow]{figures/template-flowchart.png}{系統資料流程示意圖}

\section{授權驗證流程}

使用者於請求影音內容前,必須透過 OAuth2 流程完成授權,系統方才允許傳輸。授權資訊將以 JWT Token 附加於請求標頭中,進行驗證。

\section{演算法描述}

本節展示一段簡化演算法撰寫範例,使用 `algorithmic` 環境。

\begin{algorithm}[H]
\caption{資料接收與分發流程}
\begin{algorithmic}[1]
\REQUIRE 音視頻串流資料來源
\STATE 接收並解析封包
\IF{授權驗證成功}
    \STATE 編碼後傳輸至目標客戶端
\ELSE
    \STATE 回應錯誤訊息並中止連線
\ENDIF
\end{algorithmic}
\end{algorithm}

本節所述設計可依使用者實作項目進行替換,保留方法展示與撰寫風格。
