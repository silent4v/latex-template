\chapter{相關研究}

本章節介紹以介紹串流當作範例,說明如何加入論文的參考以及cite參考到你的文章中。考慮到大家撰寫的時候會有很多資料來源,這裡分別引用了網站、技術草案、RFC規格、IEEE論文當作範例,可以查看 references.bib 看看怎麼引用

\section{即時傳輸協定與網路串流}

即時資料的穩定傳輸為視訊串流應用的核心需求之一。RTP(Real-Time Protocol)為最早被廣泛使用之標準,即由 IETF 所制定,定義於 RFC 3550 中 \cite{rfc3550}。該協定提供端對端的封包傳送與同步機制,支援時間戳記與序列編號,可有效支援即時音訊與視訊傳輸,並搭配 RTCP 進行品質監控。

在近年應用上,Haivision 提出的 SRT(Secure Reliable Transport)協定則成為替代 RTP 的新興方案 \cite{srt-protocol}。SRT 基於 UDP 實作,強化了丟包重傳、延遲修正與加密能力,並廣泛應用於不穩定網路環境下之影音串流,如遠端直播或工業監控等場景。

\section{媒體處理框架}

GStreamer 為一套成熟的開源多媒體處理框架,其設計採模組化與資料管線(pipeline)架構,支援音訊、視訊的即時處理、轉碼與混流。GStreamer 具備豐富的 Plugin 系統,由 Freedesktop.org 所維護,目前所有開發進度與原始碼皆公開於 GitLab \cite{gstreamer-gitlab},並擁有穩定之維運機制與平台支援。

本研究在實作階段中,參考其模組設計原則與 Plugin 延展方式,作為自建系統中資料流設計與模組注入的基礎概念。

\section{學術研究相關成果}

在系統效能與網路適應性方面,Hoang 等人於 IEEE Wireless Communications 所提出的動態邊緣快取架構,提供了針對使用者行為與網路條件進行快取調整的策略 \cite{hoang2018dynamic}。該架構融合機器學習模型與系統優化,證明在 5G 高密度環境中具備提升使用者體驗與頻寬利用率的潛力。

儘管本研究聚焦於低層傳輸協定與系統實作,但此類研究成果提供了系統設計時可參考之多維度考量,如節點部署位置、回應時間策略與模組間耦合性等。

\section{本章小結}

本章介紹了串流傳輸協定、媒體處理框架與近期系統最佳化之研究成果,作為後續系統設計與比較之基礎。後續章節將結合上述技術背景,闡述本研究系統之設計理念與實作方式。
